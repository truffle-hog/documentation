\chapter{Umsetzung \& Designänderungen}

\newcounter{fanr}[section]
\newcommand\fano{\arabic{fanr}}
\newcommand\fa[3]{\namedlabel{fa#1}{FA#1}: & \textit{#2: } #3 \\}

\section{Umsetzung}

In diesem Abschnitt sind alle funktionalen Anforderungen aufgezählt. Neben jeder Anforderung steht, inwiefern diese Umgesetzt wurde.

\subsection{\gls{sppname} - Funktionalität}

\def\arraystretch{1.5}
\begin{tabular}{lp{0.9\linewidth}}

\fa{20}{Erkennen von \gls{profinet} \glspl{paket}n}{Implementiert}

\fa{30}{Dekodieren von \gls{profinet} \glspl{paket}n}{Implementiert -- Pakete können teilweise dekodiert werden. Weitere Daten können leicht durch zusätzliche Subdekoder extrahiert werden. Mehr Informationen finden sich im Designheft.}

\fa{40}{Unterscheiden der verschiedenen \gls{frameid}s}{Implementiert -- \gls{frameid}s Lassen sich unterscheiden. Einige Frames werden dekodiert und weitere lassen sich bei Bedarf durch die gewählte Struktur schnell hinzufügen.}

\fa{50}{Output Kanal zu \gls{programname}}{Implementiert -- Eine \gls{interprocess} ist mittels unix sockets implementiert.}

\fa{70}{Ausgabeschnittstelle}{Implementiert -- Die Schnittstelle ist durch ein C struct gegeben.}

\end{tabular}

\subsection{\gls{sppname} - Optionale Funktionalität}

\begin{tabular}{lp{0.9\linewidth}}

\fa{80}{Markierung fehlerhafter Pakete}{Nicht Implementiert -- Fehlerhafte Pakete werden zur Zeit noch nicht markiert.}

\end{tabular}

\subsection{\gls{programname} - Funktionalität}

\begin{longtable}{lp{0.9\linewidth}}

\fa{90}{Starten von \gls{snort}}{Gestrichen -- Da \gls{snort} meistens als Service läuft und manuelle Konfiguration benötigt, wurde dieses Feature gestrichen.}

\fa{100}{\gls{sppname} anschalten}{Gestrichen -- Da \gls{snort} meistens als Service läuft und im Moment neu Kompiliert werden muss, um \gls{sppname} zu beinhalten, wurde dieses Feature gestrichen. Manuelle Konfiguration ist hier sinnvoller.}

\fa{110}{\gls{snort} läuft, aber \gls{sppname} ist deaktiviert}{Verändert implementiert -- Der Benutzer wird, aus den selben Gründen wie oben, nur darauf hingewiesen, dass das Plugin nicht über \gls{interprocess} erreichbar ist.}

\fa{120}{Empfangen von Paketinformationen}{Implementiert}

\fa{130}{Erkennen neuer Netzwerkteilnehmer}{Implementiert}

\fa{132}{Erkennen neuer Netzwerkkommunikation}{Implementiert}

\fa{133}{Erstellen neuer Kanten}{Implementiert}

\fa{135}{Erstellen neuer Knoten}{Implementiert}

\fa{140}{Erkennen von Gerätenamen}{Implementiert}

\fa{150}{Kennzeichnen illegaler Knoten}{Implementiert}

\fa{160}{Kennzeichnen illegaler Kanten}{Nicht implementiert -- Da Knoten schon gekennzeichnet werden, ist dies nicht sehr sinnvoll, da es die Übersichtlichkeit verschlechtert.}

\fa{170}{Erkennen der Kommunikationswege}{Implementiert}

\fa{180}{Paketaustausch Statistik}{Implementiert}

\fa{190}{Gesamtzahl der Pakete}{Implementiert}

\fa{220}{Zeichnen des Graphen}{Implementiert}

\fa{250}{Kantenunterschied}{Nicht implementiert -- Kanten beziehen sich nur auf ein Protokoll.}

\fa{260}{Logging}{Nicht implementiert -- Da der Netzwerkverkehr auch direkt als PCAP Datei aufgezeichnet werden kann, ist dieses Feature nicht nötig.}

\fa{270}{Spezifisches Logging}{Nicht implementiert -- Siehe \ref{fa260}}
\end{longtable}

\subsection{\gls{programname} - Optionale Funktionalität}
\begin{longtable}{lp{0.9\linewidth}}

\fa{280}{Fehlerhafte Pakete}{Nicht implementiert -- Lässt sich bei Bedarf leicht hinzufügen.}

\fa{290}{Rückverfolgung}{Teilweise implementiert -- Die Grundstruktur ist implementiert, muss aber noch überarbeitet und von Bugs befreit werden.}

\fa{300}{Kennzeichnen inaktiver Knoten}{Nicht implementiert}

\fa{305}{Darstellung des Paketvolumens}{Implementiert -- Kann nicht deaktiviert werden, da übersichtlich genug}

\fa{310}{Abgrenzen gefilterter Knoten}{Implementiert}

\fa{315}{Timeoutbenachrichtigung}{Nicht implementiert}

\fa{317}{Darstellen von \gls{snort} Alerts}{Nicht implementiert}

\end{longtable}
\subsection{\gls{programname} - Datenhaltung}

\begin{longtable}{lp{0.9\linewidth}}

\fa{320}{Speichern des Datenverkehrs}{Nicht implementiert -- Da der Datenverkehr einfach mittels PCAPs gespeichert werden kann, ist dieses Feature nicht nötig.}

\fa{330}{Kategorisierung der Daten}{Nicht implementiert -- Siehe \ref{fa320}}

\fa{340}{Datenausgabe}{Nicht implementiert -- Siehe \ref{fa320}}

\fa{350}{Zeitraum der Datenspeicherung}{Nicht implementiert -- Siehe \ref{fa320}}

\end{longtable}

\subsection{\gls{programname} - Benutzerinteraktion}

\begin{longtable}{lp{0.9\linewidth}}

\fa{360}{Speichern der Einstellungen}{Implementiert}

\fa{365}{\gls{gui}}{Implementiert}

\fa{370}{Fehlerhafte Benutzereingaben}{Implementiert}

\fa{375}{Einstellungsmenü}{Nicht implementiert -- Da keine Einstellungen vorhanden sind, wird ein Einstellungsmenü nicht benötigt.}

\fa{380}{Skalierung des Graphen}{Implementiert}

\fa{390}{Detaillierte Information zu Knoten}{Implementiert}

\fa{400}{Generelle Informationen}{Implementiert}

\fa{410}{Hover über einem Knoten}{Nicht implementiert -- Diese Funktionalität ist unübersichtlich und wurde daher nicht implementiert.}

\fa{415}{Die \gls{mac}-Addresse wird als eindeutigen identifikator für einen Knoten des Graphen benutzt.}{Implementiert}

\fa{420}{\gls{blacklist}}{Verändert implementiert -- Blacklist und Whitelist wurden vereinigt in einen allgemeinen Filter, da sie Komplementär sind}

\fa{425}{\gls{whitelist}}{Siehe \ref{fa420}}

\fa{427}{Whitelist/Blacklist aktivieren}{Verändert implementiert -- Es gibt ein Fenster mit einer Filterliste, über das man die Filter aktivieren, deaktivieren und bearbeiten kann.}

\end{longtable}

\subsection{\gls{programname} - Optionale Benutzerinteraktion}

\begin{longtable}{lp{0.9\linewidth}}

\fa{430}{Statistiken zu Knoten}{Nicht implementiert}

\fa{440}{Filtereinstellungen}{Verändert implementiert -- Es gibt ein Filtermenü, über das die Einstellungen vorgenommen werden können.}

\fa{450}{Filteroptionen}{Teilweise implementiert -- Alles, bis auf den Aktivitätsfilter wurde implementiert.}

\fa{460}{Filter deaktivieren/reaktivieren}{Verändert implementiert -- Die Filter können im Filtermenü aktiviert und deaktiviert werden. Hotkeys sind nicht sinnvoll, da der Benutzer die Möglichkeit hat, beliebig viele Filter zu erstellen.}

\fa{470}{\gls{io-supervisor} festlegen}{Nicht implementiert}

\fa{480}{Verbindungsrichtung}{Verändert implementiert -- Verbindungsrichtungen werden immer dargestellt.}

\fa{490}{Graphdarstellungsalgorithmen}{Nicht implementiert}

\fa{500}{Kategorisierung}{Verändert implementiert -- Die Filter bieten eine Möglichkeit, bestimmte Knoten zu kategorisieren.}
\end{longtable}

\section{Designänderungen}

Dieser Abschnitt behandelt nennenswerte Änderungen, die sich gegenüber dem ursprünglichen Design ergeben haben.

\subsection{Networkmodel}

Im Model wurde die Graphstruktur um eine Networkport Struktur erweitert und der Graph selbst um eine Component Struktur, die in den folgenden zwei Abschnitten näher beschrieben werden.

\subsubsection{Network-Ports}

Die ganze Netzwerkstruktur wurde um eine Abstraktionsstufe erweitert. Der Graph selbst ist nicht mehr die zentrale Schnittstelle des Netzwerks. Der Graph ist Teil der übergeordneten Netzwerkstruktur, wobei dieser lediglich für die Datenhaltung der einzelnen Knoten und Kanten verantwortlich ist.

Zusätzlich wurde diese Netzwerkstruktur mit mehreren Schnittstellen versehen die jeweils klare Aufgabenbereiche abdecken. Diese unterteilen sich in drei sogenannte Ports: ReadingPort, WritingPort und ViewPort.

Der ReadingPort dient zur spezifischen Abfrage von Informationen über den Netzwerkstatus. Zum Beispiel kann man hier einzelne Knoten anhand ihrer Adressen und Namen zurückgeben lassen. Auch kann man hier generelle Informationen über das Netzwerk erhalten wie u.a. die Anzahl an Netzwerkteilnehmern, den durchschnittliche Paketdurchsatz oder die Menge an Verbindungen.

Der WritingPort dient ausschließlich zum Schreiben neuer Informationen in das Netzwerk. Im derzeitigen Status ist dies auf das Hinzufügen von neuen Netzwerkteilnehmern und Verbindungen beschränkt.

Der ViewPort ist die Verbindung der reinen Netzwerkdaten zur Grafischen Darstellung. Somit stellt diese Schnittstelle sämtliche Informationen zur Verfügung welche auf der GUI zu sehen sein soll.
Dies beinhaltet einerseits den Graphen selbst, also die Darstellung von Knoten und Kanten. Aber auch die statistischen Informationen welche auf der GUI in Form von tabellarisch angeordneten Werten visualisiert werden.

Alles in Allem kann man die einzelnen Schnittstellen als Fassaden für die verschiedenen Verantwortungsbereiche des Programms verstehen (Einfacher und klarer zu benutzende Schnittstellen).

\subsubsection{Components}

Die Komponentenstruktur lehnt sich an das Entwurfsmuster Kompositum an. Hierbei Sind die einzelnen Bestandteile des Graphen (Knoten und Kanten) jeweils Kompositionen aus verschiedenen Komponenten die beliebig erweiterbar sind.


\subsection{NodeStatisticsUpdater}

Der NodeStatisticsUpdater ist ein Service, der neu hinzugekommmen ist. Er ist dafür Zuständig in regelmäßigen Zeitabständern (eine Sekunde) den Graph zu traversieren und
in jedem Knoten die zeitlichen Statistiken zu aktualisieren.

\subsection{Filter}

Das im Entwurfsheft festgelegte Muster für die Filter wurde komplett verworfen. Stattdessen gibt es jetzt ein Filter interface, das einen Knoten entgegennimmt, und in der
FilterPropertiesComponent entsprechende Flags setzt, je nach dem, ob der Knoten gematcht wurde, oder nicht. Zusätzlich wird bei den Filtern nicht mehr zwischen Whitelist und
Blacklist differenziert, weswegen diese jetzt eine Priorität und Farbe besitzen. So können beliebig viele Gruppierungen mittels Filtern erstellt werden. \par
Da es jetzt die Möglichkeit gibt, viele Filter gleichzeitig aktiviert zu haben, musste eine Möglichkeit geschaffen werden, diese an einer Stelle zu verwalten, sodass sie problemlos
auf alle neu erstellten Knoten angewendet werden können. Zu diesem Zweck wurde ein MakroFilter implementiert, der dem Filter Interface entspricht. Dieser Filter hat lediglich eine
Liste an Filtern und zwei Methoden, um einzelne Filter hinzufügen, oder entfernen zu können. Alle teile des Programms können nun mit diesem MakroFilter arbeiten, indem sie die check()
Methode aufrufen. Diese Methode iteriert dann durch alle internen Filter und ruft jeweils bei diesen die check() Methode auf. 