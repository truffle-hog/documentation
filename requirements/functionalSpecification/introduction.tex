\chapter*{Einleitung}

In der Industrie gewinnt Vernetzbarkeit immer mehr Bedeutung. Auch das Vernetzen von externen Geräten mit den Produktions-Maschinen wird immer gebräuchlicher, um die Arbeitseffizienz zu erhöhen. Um diese Kommunikation zu vereinfachen und verbessern wurde das Process Field Network (im Folgenden \gls{profinet}) entwickelt.\\
\gls{profinet} setzt auf Ethernet für echtzeitfähige Anwendungen und \gls{tcp}/\gls{ip} für langsamere \gls{io} Anwendungen. Durch die immer mehr vernetzten und oft dem Internet zugänglichen Produktionsstätten ist Sicherheit mittlerweile von höchster Bedeutung. \\
Dieses Projekt setzt an dieser Stelle an und soll dem \gls{ids} \gls{snort} ermöglichen, die \gls{profinet} Protokolle zu verstehen und zu verarbeiten. Um dem Benutzer eine Übersicht über die stattfindenden Kommunikationsprozesse zu verschaffen, soll eine \gls{gui} entwickelt werden. Diese soll mithilfe verschiedener Graphalgorithmen eine geordnete Darstellung ermöglichen.\\

\begin{figure}[H]
  \centering
  \includegraphics[width=300pt]{../diagrams/intro_diagram/intro_diagram.png}
  \caption{Strukturübersicht des Projekts}\label{fig:diagram}
\end{figure}

\\Das Diagramm~\ref{fig:diagram} zeigt im unteren Teil den \gls{profinet} \gls{praeprozessor}, welcher \gls{snort} die Dekodierung von \gls{profinet} Paketen ermöglicht. Die dekodierten \glspl{paket} werden per \gls{ipc} zum \gls{programname} Prozess gesendet.\\
\Gls{programname} ist nach dem \gls{mvc} Prinzip strukturiert.\\
Das \textit{MODEL} holt die Pakete von der \gls{ipc} Schnittstelle. Sobald neue Daten zur Verfügung stehen werden diese in die vorhandene Datenstruktur eingeordnet und der \textit{VIEW} wird mitgeteilt, dass \textit{neue Daten} zur Verfügung stehen. \textit{VIEW} holt daraufhin die neuen Daten und präsentiert sie entsprechend auf der \gls{gui}.\\
\textit{Benutzereingaben} im \textit{VIEW} werden dem \textit{CONTROLER} gemeldet. Dieser erzeugt \textit{Feedback} im \textit{VIEW} welches vom Benutzer wahrgenommen werden kann. Der \textit{CONTROLER} hat außerdem die Möglichkeit durch Benutzerbefehle oder andere Geschäftslogiken induzierte Manipulationen am \textit{MODEL} vorzunehmen.

 