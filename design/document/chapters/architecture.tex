\chapter{Architektur und Musterbeschreibung} 
Im Groben gehalten funktioniert der Daten- und Befehlsfluss im Trufflehog-Entwurf wie im klassischen MVC, der Presenter aktiviert Services, welche das Model basierend auf dem empfangenen Netzwerkverkehr verändert und das View aktualisiert sich am Model.\newline
\newline
\textbf{\gls{commandpkg}:} \newline
Die Kernfunktionalität innerhalb des Programms stellt das commands-package. Darin befinden sich sowohl die Hierarchie also auch jedes mögliche Command. Vom Prinzip her vergleichbar mit Runnables.\newline
Es gibt 2 Hauptarten von Commands, UserCommands für die Verwaltung der Benutzeroberflächenbefehle, und TruffleCommands für die Methodenaufrufe im Model-Graph, also die Hauptarbeit am Model.\newline
\newline
\textbf{\gls{servicepkg}:}\newline
In unserem Entwurf haben wir uns außerdem dazu entschieden einige einzelne Arbeitsschritte in das Service-Package auszulagern.\newline
 //hier Bild von Service-Package einfügen//\newline
Der truffleproessor ist die Empfangsstelle für die von uns in der IPC benutzen Truffles. Außerdem benutzt der truffleprocessor einen Generator zur Erzeugung der später verwendeten Commands.\newline
Das datalogging kümmert sich um die gesamte gewünschte Speicherung/Loggen und, falls implementiert, die Verfügbarkeit der Video-Daten. Es speichert sowohl regelmäßige Snapshots des gesamten Graphen, als auch einzelne Commands für eine Schrittweise Rückverfolgung des Ablaufes.\newline
Der executor ist das Unterpackage, in welchem die Anwendung bzw. Ausführung der Commands stattfindet. Der Presenter startet eine Instanz und stellt die CommandQueue zur Verfügung.
