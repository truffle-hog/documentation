\subsection{Receiver Executor communication}

Dieses Sequenzdiagramm beschreibt am Beispiel des CommandExecutors und des TruffleReceivers, wie das Notification Framework funktioniert.
Beide Services laufen in verschiedenen Threads. Der CommandExecutor nimmt solange Commands von seinen Queues, bis keine abzuarbeitenden
Commands mehr vorhanden sind. Falls dieser Fall eintreten sollte, so wird der Thread schlafen gelegt, bis ein neues Element auf eine
der Queues geschrieben wird. Parallel dazu empfängt der TruffleReceiver Truffle Objekte. Anschließend
wird ein AddPacketDataCommand mit einem neuen Truffle erstellt. Der Command wird dann mittels der receive Methode an den CommandExecutor listeners übergeben. Die receive Methode schreibt intern auf die CommandQueue des CommandExecutors. Dadurch wird dieser wieder aufgeweckt,
um neue Commands abzuarbeiten.

\FloatBarrier
\begin{sidewaysfigure}
  \centering
  \includegraphics[width=\textwidth]{../diagramimages/sd_receiver_executor_com.pdf}
  \caption[Sequenzdiagramm Receiver Executor communication]{Sequenzdiagramm Receiver Executor communication}
\end{sidewaysfigure} \FloatBarrier 