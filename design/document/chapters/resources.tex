\chapter{Ressourcen}
\section{Dateien und Strukturen}
Im Folgenden ist das Datenhaltungsverzeichnis von \gls{programname} visualisiert. Alle persistenten, das heißt einen Neustart des Programms überdauernden Daten sind in Dateien in Unterordnern des Ordners \texttt{data} gespeichert.
\begin{figure}[htb]
  \centering
\begin{verbatim}
.
`-- data
    |-- config
    |   |-- filtermenu_en.properties
    |   |-- filtermenu_de.properties
    |   |-- graphwindow_en.properties
    |   |-- graphwindow_de.properties
    |   |-- logwindow_en.properties
    |   |-- logwindow_de.properties
    |   |-- notificationview_en.properties
    |   |-- notificationview_de.properties
    |   |-- settingsmenu_en.properties
    |   |-- settingsmenu_de.properties
    |   |-- statistics_en.properties
    |   `-- statistics_de.properties
    |-- log
    |   `-- trufflehog.log
    |-- replay_log
    |   |-- 20151020171450.trufflesnapshot
    |   `-- 20151020171950.trufflesnapshot
    `-- truffle_data_log
        |-- node_00-0a-95-9d-68-16.xml
        `-- node_00-0a-95-9d-68-20.xml
\end{verbatim}
  \caption[Ordner- und Dateiestruktur von \gls{programname}]{Beispielhafte Ordner- und Dateiestruktur von \gls{programname}}
\end{figure}

\begin{description}[style=multiline, leftmargin=3.3cm, labelwidth=3.3cm]
\item[\texttt{config}] Einen Neustart des Programms überdauernde Informationen über die visuellen Teile des Programms und sprachliche Lokalisierungen. Diese Einstellen werden bei Programmneustart aufgerufen und \gls{programname} wird in den alten Nutzungszustand versetzt.
\item[\texttt{log}] Zentrale Logdatei des Programms. Gespeichert wird der ankommende Truffle-Netzverkehr als kurzer Text zeitlich geordnet.
\item[\texttt{replay\_log}] Gesamtzustand des Graphen zu einem Zeitpunkt plus Veränderungen über eine gewisse Zeitspanne. Realisiert ist das über die Serialisierung von sowohl eines Graph-Snapshots in regelmäßigen Abständen, als auch alle dazwischenliegenden Commands als komprimierte Liste. Wie der Name schon sagt sind diese serialisierten Objekte Kernbestand der Replay-Funktion.
\item[\texttt{truffle\_data\_log}] Logdatei zu jedem Knoten mit zeilenweiser Auflistung der verarbeiteten Truffles als Text. Dies Ermöglicht eine eventuelle Abfrage von Knoten-spezifischen Logs in der \gls{gui} und die spätere externe Analyse für den Ursprung von fehlerhaften Netzwerkpaketen.
\end{description}

\section{Externe Bibliotheken}
\subsection{Java Universal Network/Graph Framework}
Die Bibliothek Java Universal Network/Graph Framework (JUNG) ist eine seit 2003 entwickelte Java-Bibliothek zur Speicherung, Visualisierung und zu Rechnen mit Graphen und graphähnlichen Strukturen. Sie ist modular aufgebaut und kapselt alle oben genannten Funktionen in verschiedenen Bibliotheksteilen. Außerdem bietet sie erweiterte Funktionalitäten wie verschiedene Anordnungsalgorithmen, diverse Annotationsmöglichkeiten der Visualisierung und Algorithmen zur Graphanalyse.
\subsubsection{Motivation}
JUNG hat alle für \gls{programname} relevanten Funktionen einer Graphbibliothek und bietet zudem folgende Vorteile gegenüber vergleichbaren Bibliotheken.

Für sämtliche Klassen der Bibliothek werden Interfaces bereitgestellt. Dies bedeutet, dass ohne Kompatibilitätsverlust neue Funktionen eingebunden werden können und JUNG selbst mit den Klassen von \gls{programname}, welche diese Interfaces implementieren, arbeiten kann.

Die kritischen Komponenten, also die Visualisierung, Datenhaltung und Graphlogik ist in JUNG modular voneinander abgegrenzt. Dies ermöglicht die perfekte Einbindung in das \gls{mvp}-Konzept von \gls{programname}. Viele andere Graphbibliotheken koppeln Datenhaltung und Visualisierung und würden somit keine klare Trennung des \gls{mvp} ermöglichen.

JUNG ist als sehr stabile und verlässliche Bibliothek bekannt, die Wert auf Kompatibilität und weitgehende Fehlerfreiheit legt. Aufgrund der langen Entwicklungszeit und der großen Beliebtheit sind in der Benutzung, wie sie in \gls{programname} vorliegt, keine Fehler seitens der Bibliothek zu erwarten. JUNG befindet zudem noch in aktiver Entwicklung und wird auch von einer aktiven Community gepflegt.

Da JUNG ein Open Source Projekt ist kann die Fehlersuche stark vereinfacht werden und eigene Anpassungen können, falls nötig, während der Entwicklung vorgenommen werden.

\subsubsection{Verwendungsbeschreibung}

Die Verwendung von JUNG ist auf die beiden Packages \texttt{model} und  \texttt{view} beschränkt. Das Package \texttt{model} realisiert Interfaces der Packages \texttt{jung.layout} und \texttt{jung.visualization}, welche die Datenhaltung zu JUNG kompatibilisieren.

Das Package \texttt{view} realisiert den \texttt{VisualizationServer} aus \texttt{jung.visualization}, welcher die visuelle Darstellung des Graphen bereitstellt.
